\chapter{Les interactions amplifient les effets cumulatifs dans les écosystèmes marins}
\label{chap5}

\section{Résumé}

Les interactions écologiques sont un couteau à deux tranchants: les connections qui sont vitales à l’existence des communautés écologiques complexes qui ont inspiré la métaphore de Darwin sur les rivages emmêlés (“*tangled bank*”) sont également les connections qui permettent aux effets des changements climatique et des activités humains de se propager à travers les communautés. Les interactions écologiques demeurent pourtant absentes de la littérature sur la gestion environnementale. Ici, nous évaluons les effets cumulatifs des changements climatiques et des activités humains sur les espèces du Système Saint-Laurent à l’aide d’une nouvelle approche à l’échelle des réseaux qui permet de prendre en considération le réseau d’interactions qui structure les communautés. Nous comparons notre approche avec une évaluation conventionnelle à l’échelle des espèces pour exposer des propriétés émergentes provenant des interactions écologiques et trouvons des effets sur des espèces qui seraient normalement ignorés. Pour certaines espèces, considérer les interactions écologiques pourraient être l’unique moyen d’évaluer les effets des stresseurs environnementaux. Les poissons et les mammifères marins sont particulièrement susceptibles aux effets indirects de tous les types de stresseurs considérés; ceci est un contraste marqué avec le nombre limité de stresseurs qui les affectent directement. Nos résultats sont particulièrement pertinents pour la gestion d’espèces exploitées et en péril, pour qui nous ignorons potentiellement des risques importants en négligeant les effets indirects provenant des interactions entre espèces.

Ce cinquième article, intitulé \textit{"Interactions amplify cumulative effects in marine systems"} a été corédigé par moi-même, Kevin Cazelles, Rémi M. Daigle, Dominique Gravel et Philippe Archambault. Cet article est le résultat de l'entièreté de la thèse. Il combine les travaux présentés aux chapitres précédents et propose des analyses supplémentaires en vue d'effectuer une évaluation des effets cumulatifs communauté-centrée pour le Système du Saint-Laurent. J'ai structuré et établi les objectifs de cet article conjointement avec Dominique Gravel et Philippe Archambault. J'ai structuré et formaté les données, effectué les analyses et été en charge des développements techniques. Kevin Cazelles et Rémi M. Daigle ont contribué à ces étapes. J'ai dirigé la rédaction de l'article. Tous les co-auteurs ont contributé à la rédaction et aux révisions de l'article Considérant l'originalité des analyses proposée et de la méthodologie développée, j'ai espoir que cet article sera intéressant pour une revue généraliste. C'est la raison pour laquelle cet article est dans un format court, avec la méthodologie présentée à la fin de l'article. \linebreak[4]


\textit{Les sections suivantes sont celles de l'article en préparation.}
