\chapter{Caractériser et partager les connaissances sur l’exposition aux stresseurs environnementaux dans le Système du Saint-Laurent au Canada}
\label{chap4}

\section{Résumé}

Le Système du Saint-Laurent est un système socio-écologique vaste et complexe qui soutient une myriade de secteurs économiques. Cet écosystème est affecté par plusieurs pressions humaines qui se chevauchent et qui peuvent interagir avec les effets des changements climatiques. L’objectif de cet article était de caractériser la distribution et l’intensité des stresseurs environnementaux dans le Système du Saint-Laurent. Nous avons rassembler des données sur 22 stresseurs d’origine côtière, du climat, de la pêche et du trafic maritime à partir de collaborations, d’initiatives environnementales existantes et de portails de données ouvertes. Nous montrons que peux de milieux du Saint-Laurent sont libres d’exposition cumulée. L’estuaire, la gyre d’Anticosti, et les milieux côtiers sont particulièrement exposés, surtout près des centres urbains. Nous avons identifié 6 groupes distincts identifiant des régions exposées à des stresseurs similaires. Nous montrons que différentes combinaisons de stresseurs sont typiques de certaines régions du Saint-Laurent et que les milieux côtiers sont exposés à tous les types de stresseurs. Deux groupes particulièrement préoccupants capturent la majorité des points chauds d’exposition cumulée et montrent une convergence de groupes de stresseurs différents à la tête du Chenal Laurentien. Le partage des connaissances acquises sur les stresseurs environnementaux dans le Saint-Laurent est devenu une priorité dans le cadre de ce projet. C’est pourquoi nous initions \textit{eDrivers}, une plateforme de connaissances ouvertes qui réunit des experts commis à structurer, standardiser et partager les connaissances sur les stresseurs environnementaux en support à la science et la gestion holistique. \textit{eDrivers} a été construit selon des principes directeurs visant à soutenir les standards de gestion des données et de science ouverte existants. Nous anticipons ainsi que l’initiative améliorera graduellement les connaissances qui y sont partagées. Nous croyons que \textit{eDrivers} est une solution indispensable qui pourra influencer radicalement la recherche et la gestion à large échelle en accroissant l’accessibilité et l’interopérabilité des connaissances sur les stresseurs environnementaux.

Ce quatrième article, intitulé \textit{"Characterizing exposure to and sharing knowledge of drivers of environmental change in the St. Lawrence System in Canada"} a été corédigé par moi-même, Rémi M. Daigle, Steve Vissault, Dominique Gravel, Andréane Bastien, Simon Bélanger, Pascal Bernatchez, Marjolaine Blais, Hugo Bourdages, Clément Chion, Peter S. Galbraith, Benjamin S. Halpern, Camille Lavoie, Christopher W. McKindsey, Alfonso Mucci, Simon Pineault, Michel Starr, Anne-Sophie Ste-Marie et Philippe Archambault. Il a été publié dans la revue \textit{Frontiers in Marine Science} au sein de la section spéciale \textit{Global Change and the Future Ocean} à l'été 2020. J'ai établi les objectifs et la structure de l'article avec Rémi M. Daigle, Dominique Gravel et Philippe Archambault. J'ai structuré et formaté les données, effectué les analyses, été en charge des développements techniques et dirigé la rédaction de l'article. Tous les co-auteurs ont contributé aux données, aux analyses et à la rédaction selon leur expertise respective et contribué à la révision de l'article. Les résultats issus de cet article ont été présentés en version abrégée lors de la \textit{Réunion scientifique annuelle de Québec Océan} à Rivière-du-Loup (Canada) à l'automne 2017 et à la \textit{Réunion annuelle du regroupement de recherche Canadian Healthy Oceans Network (CHONe)} à Ottawa (Canada) à l'hiver 2018. \linebreak[4]


\begin{singlespace}
  Beauchesne, D., Daigle, R.M., Vissault, S., Gravel, D., Bastien, A., Bélanger, S., Bernatchez, P., Blais, M., Bourdages, H., Chion, C., Galbraith, P.S., Halpern, B.S., Lavoie, C., McKindsey, C.W., Mucci, A., Pineault, S., Starr, M., Ste-Marie, A.-S., Archambault, P., 2020. Characterizing Exposure to and Sharing Knowledge of Drivers of Environmental Change in the St. Lawrence System in Canada. Front. Mar. Sci. 7. https://doi.org/10.3389/fmars.2020.00383
\end{singlespace}

\textit{Les sections suivantes sont celles de l'article publié.}
