With demands for natural resources increasing alongside populations and
the effects of climate change intensifying, ecosystems worldwide are
increasingly burdened with the cumulative effects of a vast array of
environmental stressors. These observations stimulate a growing demand
for ecosystem-based approaches and regional cumulative effects
assessments. Yet environmental management still overwhelmingly operates
in silos, focusing instead on single-stressor and single-species
assessments. This is particularly concerning for exploited and
endangered species whose dynamics, like that of all species, is driven
by the network of interactions structuring ecological communities and
through which the effects of stressors can propagate indirectly. A
general lack of theoretical understanding of the effects of multiple
stressors on ecological communities, and methodological and logistical
constraints explain part of this discrepancy. In this context, the
general objective of this thesis is to evaluate the cumulative effects
of climate change and human activities on the ecological communities of
the St.~Lawrence System in eastern Canada. The general hypotheses
guiding the thesis are that species interactions influence the indirect
and non-additive spread of the effects of multiple stressors through
communities and that, as such, species interactions and stressors should
be considered together in network-scale cumulative effects assessments.

The thesis is divided in three parts. In the first part (chapter 1), I
conceptualize how stressors propagate through food webs and explore how
they affect simulated 3-species motifs and food webs of the Canadian
St.~Lawrence System. We find that overlooking species interactions
invariably underestimates the effects of stressors, and that synergistic
and antagonistic effects through food webs are prevalent. At the scale
of food webs, we find that apex predators generally were negatively
affected and mesopredators benefited from the effects of stressors in
the St.~Lawrence System, but that species sensitivity is dependent on
food web structure. The theoretical simulations proposed in chapter 1
validate the general hypotheses of the thesis and provide an accessible
and theory-grounded framework for the inclusion of species interactions
in cumulative effects assessments.

The second part of the thesis addresses logistical and methodological
challenges for the cumulative effects assessment of climate change and
human activities on communities of the St.~Lawrence System. In chapter
2, I address the challenge of characterizing ecological interactions in
data-deficient ecosystems. I present a new unsupervised machine learning
method to predict interactions between any given set of taxa, given
pairwise taxonomic proximity and known consumer and resource sets
available through various open-data portals. Results from chapter 2
suggest that ecological interactions can be predicted with high
accuracy, which could promote their use for environmental management. In
chapter 3, I identify environmental issues and suggest research and
management priorities to promote cumulative effects assessment and
ecosystem-based management in the St.~Lawrence System. In chapter 4, I
characterize the distribution and intensity of environmental stressors
arising from human activities and climate change in the St.~Lawrence
System. Through collaborations, existing environmental initiatives and
open data portals, I gathered data-based indicators for 22 coastal,
climate, fisheries, and marine stressors. Results from chapter 4 show
that stressors are widespread and that coastal areas and the Estuary,
Anticosti Gyre, and coastal areas are particularly exposed to cumulative
exposure and hotspots.

In the third part of the thesis (chapter 5) I present a new
network-scale approach to assess cumulative effects that explicitly
considers ecological interactions and indirect effects. The approach is
built on the framework presented in chapter 1 and uses results and
methods from chapters 2, 3 and 4. I compare our approach to a
conventional species-scale assessment to expose transgressive properties
arising from species interactions and uncover cumulative effects to
species that would otherwise be overlooked. Fishes and marine mammals
appear particularly prone to indirect effects from all types of
stressors; this contrasts considerably with the limited number of
stressors affecting them directly. For certain species, considering
interactions may even be the only means of assessing the effects of
stressors.

My research findings show that the intricacies of ecological communities
are key to assess the direct and indirect effects of multiple stressors
on species and how ecological interactions can be explicitly considered
in cumulative effects assessment. This is particularly relevant to the
management of exploited and endangered species for which we may
currently ignore significant threats by overlooking the less obvious yet
no less significant effects arising from species interactions. My thesis
also promotes a systems mindset that could be instrumental in fulfilling
the promise of ecosystem-based management.

\begin{quote}
Keywords: cumulative effects, environmental stressors, indirect effects,
biotic interactions, non-additive effects, ecological network, food web,
St.~Lawrence System, human activities, climate change.
\end{quote}
