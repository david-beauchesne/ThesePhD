La demande croissante en ressources naturelles et l'intensification des
changements climatiques, les écosystèmes sont de plus en plus soumis aux
effets cumulatifs d'une variété de stresseurs environnementaux. Ces
observations stimulent une demande croissante pour les approches de
gestion écosystémique et les évaluations régionales des effets effets
cumulatifs. Pourtant, une gestion environnementale par silo centrée sur
les évaluations par espèces et stresseurs individuels demeure la norme.
Cette absence d'approches holistiques est particulièrement inquiétante
pour la gestion d'espèces exploitées ou en péril puisque leur dynamique,
comme celle de toutes espèces, est régie par le réseau d'interactions
liant les espèces entre elles au sein d'une communauté écologique et à
travers lequel les effets des stresseurs peuvent se propager
indirectement. Cette disparité entre les besoin d'approches holistiques
et la pratique est partiellement expliqué par un manque généralisé de
connaissances théorique de l'effets de multiple stresseurs sur les
communautés écologiques et par des contraintes logistiques et
méthodologiques. Dans ce contexte, l'objectif général de ma thèse est
d'évaluer les effets cumulatifs des changements climatiques et des
activités humaines sur les communautés écologiques du Système du
Saint-Laurent au Canada. Les hypothèses générales de la thèse sont que
les interactions influencent la propagation indirecte et non-additive
des effets de multiples stressors à travers les communautés et que, en
tant que tel, les interactions et les stresseurs devraient être
considérés conjointement au sein d'analyse des effets cumulatifs
communauté-centrée.

La thèse est divisée en trois partie. Dans la première partie (chapitre
1), je conceptualise la propagation des stresseurs environnementaux à
travers les réseaux trophiques et j'explore comment des stresseurs
simulés affectent des motifs à trois espèces et des communautés du
Système Saint-Laurent. Nous trouvons que négliger les interactions
écologiques sous-estime systématiquement les effets des stresseurs et
que les effets synergiques et antagonistes sont fréquents à l'échelle
des réseaux trophiques. À l'échelle des réseaux, nous trouvons que les
prédateurs apicaux étaient négativement affectés, alors que les
méso-prédateurs bénéficiaient des effets des stresseurs dans le Système
Saint-Laurent, mais que la sensibilité des espèces dépend de la
structure des réseaux trophiques. Le travail théorique proposé au
chapitre 1 permet de valider les hypothèses générale de la thèse et
offre un cadre accessible et appuyé par la théorie écologique pour
inclure les interactions écologiques à l'évaluation des effets
cumulatifs.

La deuxième partie de ma thèse aborde des défis logistiques et
méthodologiques pour l'évaluation des effets cumulatifs des changements
climatiques et des activités humaines sur les communautés du Système du
Saint-Laurent. Au chapitre 2, j'aborde le défi de caractériser les
interactions écologiques au sein de systèmes où peu de données sont
disponibles. Je présente une nouvelle méthode d'apprentissage non
supervisée pour prédire les interactions binaires sachant la proximité
taxonomique entre taxa et les ensembles de consommateurs et de
ressources connus et provenant de plusieurs portails de données
ouvertes. Les résultats suggèrent que les interactions écologiques
peuvent être prédites avec une forte précision, ce qui pourrait
promouvoir leur utilisation pour la gestion environnementale. Au
chapitre 3, j'identifie des enjeux environnementaux et suggère des
priorités de recherche et de gestion pour promouvoir les évaluations
d'effets cumulatifs et la gestion écosystémique dans le Système du
Saint-Laurent. Au chapitre 4, je caractérise la distribution et
l'intensité de stresseurs environnementaux issus des activités humaines
et des changements climatiques dans le Système du Saint-Laurent. À
partir de collaborations, des initiatives environnementales existantes
et des portails de données ouvertes, j'ai amassé des données sur 22
stresseurs d'origine côtière, du climat, de la pêche et du trafic
maritime. Les résultats du chapitre 4 démontrent que les stresseurs sont
répandus partout dans le Système du Saint-Laurent et que l'estuaire, la
gyre d'Anticosti, et les milieux côtiers sont particulièrement exposés
aux stresseurs.

Dans la troisième partie de la thèse (chapitre 5), je présente une
nouvelle approche d'évaluation des effets cumulatifs communauté-centrée
qui considère les interactions écologiques et les effets indirects.
L'approche proposée s'appuie sur le cadre théorique présenté au chapitre
1 et utilise les résultats et méthodes des chapitres 2, 3 et 4. Je
compare cette approche à une approche conventionnelle à l'échelle des
espèces pour exposer des propriétés émergentes provenant des
interactions écologiques et trouvons des effets sur des espèces qui
seraient normalement ignorés. Pour certaines espèces, considérer les
interactions écologiques pourraient être l'unique moyen d'évaluer les
effets des stresseurs environnementaux. Les poissons et les mammifères
marins sont particulièrement susceptibles aux effets indirects de tous
les types de stresseurs considérés; ceci est un contraste marqué avec le
nombre limité de stresseurs qui les affectent directement.

Ma thèse montre que la complexité des communautés écologique est un
élément clé permettant l'évaluation des effets directs et indirects de
multiples stresseurs et comment les interactions écologiques peuvent
être intégrées aux évaluations des effets cumulatifs. Mes résultats sont
particulièrement pertinents pour la gestion d'espèces exploitées et en
péril, pour qui nous ignorons potentiellement des risques importants en
négligeant les effets indirects provenant des interactions entre
espèces. Ma thèse valorise également une pensée systémique qui est
essentielle à la gestion écosystémique.

\begin{quote}
Mots clés: effets cumulatifs, stresseurs environnementaux, effets
indirects, interactions biotiques, effets non-additifs, communautés
écologiques, réseaux trophiques, Système du Saint-Laurent, activités
humaines, changements climatiques
\end{quote}
