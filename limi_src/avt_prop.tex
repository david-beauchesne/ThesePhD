Mon projet de doctorat s'insère dans le Canadian Healthy Oceans Network
(CHONe) du CRSNG. CHONe avait deux thèmes principaux de recherche, soit
les stratégies de conservation des écosystèmes marins ainsi que
l'identification des principaux stresseurs, incluant les effets
cumulatifs, qui altèrent la biodiversité marine et les fonctions et
services écosystémiques. Mon projet était à l'origine un projet de
maîtrise sur l'identification d'indicateurs de conditions benthiques à
l'échelle du Golfe du Saint-Laurent et devait mener à la caractérisation
des principales sources de stress au sein du Saint-Laurent. En fonction
de mes intérêts de recherche, nous avons développé ce projet de maîtrise
en un projet de doctorat qui vise également l'intégration des
interactions écologiques pour l'évaluation des effets cumulatifs directs
et indirects des stresseurs environnementaux sur les communautés
écologiques.

Mes travaux de thèse ont mené à la préparation de 5 articles
scientifiques, dont 3 sont publiés, 1 est en révision et le dernier est
en préparation. J'ai également contribué à la rédaction d'un chapitre de
livre sur l'évaluation des effets cumulatifs au sein du Système du
Saint-Laurent. J'ai également présenté mes travaux de recherche à de
multiples conférences à travers 11 présentations orales et 10 affiches
scientifiques.

Je veux reconnaître les nombreuses sources de support financier et
scientifique dont j'ai bénéficié tout au long de ma thèse. Je débute en
remerciant les organismes subventionnaires qui ont rendu mon projet
possible. Je remercie le Conseil de recherches en sciences naturelles et
en génie du Canada (CRSNG), qui m'a octroyé une bourse d'études
supérieures de doctorat (ES D), et le Fonds de recherche du Québec --
Nature et Technologies (FRQNT), qui m'a octroyé une bourse de doctorat
en recherche.

Je tiens également à remercier les multiples regroupements de recherche
dont j'ai été membre pendant ma thèse et qui m'ont fourni du support
scientifique et financier, ainsi que des expériences inoubliables et
enrichissantes autant d'un point de vue personnel que professionnel.
Merci à Québec Océan, le Centre de la science de la biodiversité (CSBQ),
le programme Computational Biodiversity Science and Services
(BIOS\(^2\)), Notre Golfe et Takuvik. Je remercie également le
regroupement CHONe et ses partenaires, soit le Département Pêches et
Océans Canada et l'INREST en tant que représentant du Port de Sept-Îles
et de la Ville de Sept-Îles. Je remercie également l'Observatoire Global
du Saint-Laurent (OGSL) pour son support technique dans l'établissement
de la plateforme \emph{eDrivers} et pour la collaboration prévue qui ira
au-delà de ma thèse.

Je remercie également divers ministères provinciaux et fédéraux qui ont
contribué au projet, que ce soit par du support financier, du temps
d'experts ou des données. Un merci particulier à Pêches et Océans
Canada, sans qui ce projet n'aurait été possible. Merci également au
Ministère de l'Environnement et de la Lutte contre les changements
climatiques (MELCC) et au Ministère de l'Agriculture, des Pêcheries et
de l'Alimentation du Québec (MAPAQ) du Gouvernement du Québec, au
Département Agriculture, Aquaculture et Pêches du Gouvernement du
Nouveau-Brunswick, au Département de Pêcheries et de l'Aquaculture du
Gouvernement de la Nouvelle-Écosse, et au Département de Pêcheries, de
Foresterie et d'Agriculture du Gouvernement de Terre-Neuve-et-Labrador.

Finalement, je tiens à remercier l'Université du Québec à Rimouski,
l'Université Laval et l'Université de Sherbrooke pour m'avoir accueilli,
fournit des locaux de travail et du support administratif.
