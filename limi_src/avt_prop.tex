Mon projet de doctorat s'insère dans le Canadian Healthy Oceans Network
(CHONe) du CRSNG. CHONe avait deux thèmes principaux de recherche, soit
les stratégies de conservation des écosystèmes marins ainsi que
l'identification des principaux stresseurs, incluant les effets
cumulatifs, qui altèrent la biodiversité marine et les fonctions et
services écosystémiques. Mon projet était à l'origine un projet de
maîtrise sur l'identification d'indicateurs de conditions benthiques à
l'échelle du Golfe du Saint-Laurent et devait mener à la caractérisation
des principales sources de stress au sein du Saint-Laurent. En fonction
de mes intérêts de recherche, nous avons développé ce projet de maîtrise
en un projet de doctorat qui vise également l'intégration des
interactions écologiques pour l'évaluation des effets cumulatifs directs
et indirects des stresseurs environnementaux sur les communautés
écologiques.

Mes travaux de thèse ont mené à la préparation de 5 articles
scientifiques, dont 3 sont publiés, 1 est en révision et le dernier est
en préparation. J'ai également contribué à la rédaction d'un chapitre de
livre sur l'évaluation des effets cumulatifs au sein du Système du
Saint-Laurent et un article de vulgarisation scientifique. J'ai
également présenté mes travaux de recherche à de multiples conférences à
travers 11 présentations orales et 9 affiches scientifiques. Une liste
complète des publications et conférences est disponible à la fin de
l'avant-propos.

Je veux reconnaître les nombreuses sources de support financier et
scientifique dont j'ai bénéficié tout au long de ma thèse. Je débute en
remerciant les organismes subventionnaires qui ont rendu mon projet
possible. Je remercie le Conseil de recherches en sciences naturelles et
en génie du Canada (CRSNG), qui m'a octroyé une bourse d'études
supérieures de doctorat (ES D), et le Fonds de recherche du Québec --
Nature et Technologies (FRQNT), qui m'a octroyé une bourse de doctorat
en recherche.

Je tiens également à remercier les multiples regroupements de recherche
dont j'ai été membre pendant ma thèse et qui m'ont fourni du support
scientifique et financier, ainsi que des expériences inoubliables et
enrichissantes autant d'un point de vue personnel que professionnel.
Merci à Québec Océan, le Centre de la science de la biodiversité (CSBQ),
le programme Computational Biodiversity Science and Services
(BIOS\(^2\)), Notre Golfe et Takuvik. Je remercie également le
regroupement CHONe et ses partenaires, soit le Département Pêches et
Océans Canada et l'INREST en tant que représentant du Port de Sept-Îles
et de la Ville de Sept-Îles. Je remercie également l'Observatoire Global
du Saint-Laurent (OGSL) pour son support technique dans l'établissement
de la plateforme \emph{eDrivers} et pour la collaboration prévue qui ira
au-delà de ma thèse.

Je remercie également divers ministères provinciaux et fédéraux qui ont
contribué au projet, que ce soit par du support financier, du temps
d'experts ou des données. Un merci particulier à Pêches et Océans
Canada, sans qui ce projet n'aurait été possible. Merci également au
Ministère de l'Environnement et de la Lutte contre les changements
climatiques (MELCC) et au Ministère de l'Agriculture, des Pêcheries et
de l'Alimentation du Québec (MAPAQ) du Gouvernement du Québec, au
Département Agriculture, Aquaculture et Pêches du Gouvernement du
Nouveau-Brunswick, au Département de Pêcheries et de l'Aquaculture du
Gouvernement de la Nouvelle-Écosse, et au Département de Pêcheries, de
Foresterie et d'Agriculture du Gouvernement de Terre-Neuve-et-Labrador.

Finalement, je tiens à remercier l'Université du Québec à Rimouski,
l'Université Laval et l'Université de Sherbrooke pour m'avoir accueilli,
fournit des locaux de travail et du support administratif.

\textbf{Publications scientifiques:}

\begin{singlespace}

Beauchesne, D, Cazelles, K, Daigle, R M, Gravel, D, Archambault, P (In preparation). Ecological interactions amplify cumulative effects in marine ecosystems.

Beauchesne, D, Cazelles, K, Archambault, P, Dee, L, Gravel, D, (In revision). On the sensitivity of food webs to multiple stressors (preprint). Preprints. https://doi.org/10.22541/au.159621485.58777803

Beauchesne, D, Daigle, R M, Vissault, S, Gravel, D, Bastien, A, Bélanger, S, Bernatchez, P, Blais, M, Bourdages, H, Chion, C, Galbraith, P S, Halpern, B S, Lavoie, C, McKindsey, C W, Mucci, A, Pineault, S, Starr, M, Ste-Marie, A-S, Archambault, P (2020). Characterizing Exposure to and Sharing Knowledge of Drivers of Environmental Change in the St. Lawrence System in Canada. Frontiers in Marine Science, 7. https://doi.org/10.3389/fmars.2020.00383.

Carrier-Belleau C, Beauchesne D, Dreujou E, Isabel L (2019) Le Saint-Laurent : comment évaluer l’empreinte humaine dans un système complexe ? L'interdisciplinaire, Institut Hydro-Québec en environnement, développement et société (Institut EDS), Université Laval, Numéro 17, Automne 2019

Schloss IR, Archambault P, Beauchesne D, Bourgault D, Cusson M, Dumont D, Ferreyra G, Levasseur M, Pelletier É, St-Louis R, Tremblay R (2018) Impacts potentiels cumulés des facteurs de stress liés aux activités humaines sur l’écosystème marin du Saint-Laurent. Notre Golfe network.

Beauchesne D, Desjardins-Proulx P, Archambault P, Gravel D (2016) Thinking outside the box – predicting interactions in data-poor environments. Life and Environment. 66(3-4):333-342

Beauchesne D, Grant C, Gravel D, Archambault P (2016) L’évaluation des impacts cumulés dans l’estuaire et le golfe du Saint-Laurent : vers une planification stratégique de l’utilisation et de l’exploitation des ressources. Le Naturaliste Canadien. 140(2):45-55

\end{singlespace}

\textbf{Communications orales et par affiches lors de conférences:}

\begin{singlespace}

Beauchesne D (2019) Évaluation de l’exposition cumulée au sein du Saint-Laurent: défis, solutions et perspectives. Invited speaker. Institut Maurice Lamontagne, Department of Fisheries and Oceans Canada, Mont-Joli, Québec, Canada. May 2nd. (présentation orale)

Beauchesne D, Cazelles K, Archambault P, Gravel D (2019) La sensibilité des réseaux trophiques à de multiples perturbations. Forum québécois en sciences de la mer, Rimouski, Québec, Canada, November 11-13th. (présentation par affiche)

Beauchesne D, Daigle R, Goldsmit J, Metaxas A, Snelgrove P, Archambault P (2018) An overview of activities and outputs of the mentoring program for the 4th World Conference on Marine Biodiversity. CHONe II 2018 annual meeting, Ottawa, Ontario, Canada, November 29th - December 1st. (présentation orale)

Beauchesne D, Daigle R, Vissault S, Gravel D, Bastien A, Bélanger S, Bernatchez P, Chion C, Galbraith PS, Halpern BS, Lavoie C, McKindsey CW, Mucci A, Starr M, Ste-Marie A-S, Archambault P (2018) Next Generation Planning - Structuring and Sharing Environmental Drivers Data for the St. Lawrence. CHONe II 2018 annual meeting, Ottawa, Ontario, Canada, November 29th - December 1st. (présentation par affiche)

Beauchesne D, Daigle R, Vissault S, Gravel D, Bélanger S, Bernatchez P, Chion C, Galbraith PS,Halpern BS, McKindsey CW, Mucci A, Starr M, Archambault P (2018) Regional assessment of cumulative impacts in the St. Lawrence system. CHONe II 2018 annual meeting, Ottawa, Ontario, Canada, November 29th - December 1st. (présentation par affiche)

Isabel L, Beauchesne D, Daigle R, Archambault P (2018) CHONe project presentation. CHONe II 2018 annual meeting, Ottawa, Ontario, Canada, November 29th - December 1st. (présentation orale)

Beauchesne D, Daigle R, Vissault S, Lavoie C, Gravel D, Archambault P (2018) Meta-networks Using network theory to structure and evaluate cumulative impacts. 4th World Conference on Marine Biodiversity. Montréal, Québec, Canada. May 13-15th. (présentation orale)

Beauchesne D, Gravel D, Archambault P (2018) Next Generation Planning - Evaluating cumulative impacts on the communities of the Estuary and Gulf of St. Lawrence. Invited speaker at Guelph University. Guelph, Ontario, Canada. February 20th (présentation orale)

Beauchesne D, Daigle R, Vissault S, Gravel D, Bélanger S, Bernatchez P, Chion C, Galbraith PS,Halpern BS, McKindsey CW, Mucci A, Starr M, Archambault P (2018) Évaluation régionale des impacts cumulés sur le système du Saint-Laurent. Québec Ocean's Annual Scientific Meeting 2018, Rivière-du-Loup, Québec, Canada. November 5-6th. (présentation par affiche)

Beauchesne D, Cazelles K, Blanchet F G, Gravel D, Archambault P (2017) Predicting the spatial distribution of ecological networks. Canadian Society for Ecology and Evolution Meeting 2017, Victoria, British-Columbia, Canada. May 7-11th. (présentation par affiche)

Beauchesne D, Daigle R, Gravel D, Mckindsey C, Therriault T, Archambault P (2017) Indicators of benthic condition at the gulf-scale: megabenthic community structure. CHONe II 2017 annual meeting, Gatineau, Québec, Canada, May 1-5th. (présentation orale)

Beauchesne D, Gravel D, Archambault P (2017) Cumulative impacts on the communities of the estuary and the gulf of St. Lawrence. CHONe II 2017 annual meeting, Gatineau, Québec, Canada, May 1-5th. (présentation par affiche)

Beauchesne D, Desjardins-Proulx P, Archambault P, Gravel D (2017) Using machine learning to predict biotic interactions. Guest lecturer, Midi numérique, Biology Department, Université de Sherbrooke, Sherbrooke, Québec, Canada. February 27th 2017. (présentation orale)

Beauchesne D (2017) An “homage” to 50 years of cumulative effects assessment or Why are we still not managing resources systematically? Guest lecturer, Advanced Seminar in Environmental Science, Department of Geography, Planning and Environment, Concordia University, Montréal, Québec. February 1st 2017. (présentation orale)

Beauchesne D, Daigle R, Gravel D, Bélanger S, Bernatchez P, Chion C, Galbraith PS, Halpern B, Lavoie C, McKindsey C, Starr M, Tremblay C, Vissault S, Archambault P (2017) Next Generation Planning - Structuring and Sharing Stressors Data for the St. Lawrence. Québec Ocean's Annual Scientific Meeting 2017, Rivière-du-Loup, Québec, Canada. November 7-9th. (présentation par affiche)

Beauchesne D, Desjardins-Proulx P, Archambault P, Gravel D (2016) Thinking outside the box – predicting biotic interations in data-poor environments. Annual meeting of the Québec Center for Biodiversity Science (QCBS), Montreal, Québec, Canada, December 15th-16th. (présentation orale)

Beauchesne D, Desjardins-Proulx P, Archambault P, Gravel D (2016) The link between machine learning, networking, pokemon and ecology. Québec Océans’s 15th Annual General Assembly, Rimouski, Canada, 8-9th November 2016 (présentation orale)

Beauchesne D, Gravel D, Archambault P (2015) Évaluation des impacts cumulés sur les communautés benthiques et pélagiques de l’estuaire et du golfe du Saint-Laurent. Québec-Océan's 14th Annual General Assembly, Québec, Québec, Canada, November 10th-11th. DOI : 10.13140/RG.2.1.2209.8961 (présentation par affiche)

Beauchesne D, Gravel D, Archambault P (2015) Évaluation des impacts cumulés sur les communautés de l’estuaire et du golfe du Saint-Laurent. Annual meeting of the Québec Center for Biodiversity Science (QCBS), Montreal, Québec, Canada, October 29th-30th. (présentation orale)

Beauchesne D, Gravel D, Archambault P (2015) Évaluation des impacts cumulés sur les communautés de l’estuaire et du golfe du Saint-Laurent. Annual meeting of the Québec Center for Biodiversity Science (QCBS), Montreal, Québec, Canada, October 29th-30th. (présentation par affiche)

\end{singlespace}
