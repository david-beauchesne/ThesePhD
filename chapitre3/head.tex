\chapter{L'évaluation des impacts cumulés dans l'estuaire et le golfe du Saint-Laurent : vers une planification systémique de l'exploitation des ressources}
\label{chap3}

\section{Résumé}

L'intensification de l'empreinte humaine dans l'estuaire et le golfe du Saint-Laurent impose une planification systémique de l'exploitation des ressources marines. Une évaluation régionale des impacts cumulés dans le Saint-Laurent demeure pourtant encore attendue. Un nombre important d'activités (p. ex. transport maritime, pêche, aquaculture) caractérise l'exploitation humaine du Saint-Laurent. Ces activités imposent plusieurs stresseurs environnementaux (p. ex. destruction de l'habitat) affichant un chevauchement spatial croissant. Individuellement, ils peuvent affecter la structure et le fonctionnement des écosystèmes. Imposés simultanément, les stresseurs peuvent agir en synergie et entraîner des effets non linéaires imprévisibles. Ces effets demeurent largement incompris et conséquemment ignorés lors d'évaluations d'impacts environnementaux, qui demeurent orientées sur des espèces ou secteurs uniques et l'approbation de projets. Plusieurs défis relatifs aux impacts cumulés dans le Saint-Laurent doivent être relevés : 1) améliorer l'état des connaissances des impacts de multiples stresseurs sur les écosystèmes, 2) améliorer l'applicabilité des méthodes d'évaluation d'impacts cumulés, 3) identifier des indicateurs d'impacts cumulés, 4) créer un protocole de suivi environnemental et d'impacts humains, et de partage de données et 5) développer une capacité de gestion adaptative pour le Saint-Laurent. La planification systémique de l'utilisation des ressources naturelles au sein du Saint-Laurent nécessitera une vision intégrative de la structure et du fonctionnement des écosystèmes ainsi que des vecteurs de stress qui leur sont imposés. Une telle approche ne sera réalisable que lorsque nous aurons développé les infrastructures et les outils nécessaires à une gestion écosystémique du Saint-Laurent.

Ce troisième article a été corédigé par moi-même, Cindy Grant, Dominique Gravel et Philippe Archambault. Il a été publié dans un numéro spécial de la revue \textit{Le Naturaliste Canadien} sur le Saint-Laurent à l'été 2016. Tous les auteurs ont contribué à l'élaboration des objectifs de l'article et à sa rédaction. J'ai été en charge des figures et j'ai dirigé la rédaction de l'article. Tous les auteurs ont contribué à l'écriture et à la révision de l'article. \linebreak[4]

\begin{singlespace}
  Beauchesne, D., Grant, C., Gravel, D., Archambault, P., 2016. L'évaluation des impacts cumulés dans l'estuaire et le golfe du Saint-Laurent : vers une planification systémique de l'exploitation des ressources. Le Naturaliste Canadien 140, 45-55. https://doi.org/10.7202/1036503ar.
\end{singlespace}

\textit{Les sections suivantes sont celles de l'article publié.}
