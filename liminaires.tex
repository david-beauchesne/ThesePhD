%----------------------------------------------------------------------%
% Liminaires de la thèse.                                              %
% UQAR septembre 2013                                                  %
% ---------------------------------------------------------------------%

% ----------------------------------------------------------------------%
% 1- Page titre.                                                        %
% ----------------------------------------------------------------------%

\Pagetitre
\cleardoublepage
% ----------------------------------------------------------------------%
% inclusions qui pourraient mériter d'être incluses dans le .cls
% (commentez si non-nécessaire)
% 1.1 - Composition du Jury.                                           %
\thispagestyle{empty}

\null
\vfill
\noindent \textbf{Composition du jury:}\\
\vspace{1cm}

\begin{singlespace}
  \noindent \textbf{Fanny Noisette, présidente du jury, Université du Québec à Rimouski}\\

  \noindent \textbf{Philippe Archambault, directeur de recherche, Université Laval}\\

  \noindent \textbf{Dominique Gravel, codirecteur de recherche, Université de Sherbrooke}\\

  \noindent \textbf{Jean-Claude Brêthes, examinateur interne, Université du Québec à Rimouski}\\

  \noindent \textbf{Isabelle Côté, examinatrice externe, Université Simon Fraser}\\
\end{singlespace}

\vspace{2cm}
\noindent Dépôt initial le 25 septembre 2020
\hspace{3cm}
\noindent Dépôt final le 10 décembre 2020


\cleardoublepage

% % 1.2 - Avertissement biblio.
\input{limi_src/avertissement.tex}
% % 1.3 - Dedicace.
\thispagestyle{empty}

\begin{minipage}[l]{0.45\textwidth}

\end{minipage}%
\hfill
\begin{minipage}[r]{0.5\textwidth}
\begin{quotation}
\begin{doublespace}

à Catherine, Liam et Alyssa,

\bigskip

à la mémoire de Charles, Yvette et Thérèse


\end{doublespace}
\end{quotation}
\end{minipage}%

\cleardoublepage

% ----------------------------------------------------------------------%


% ----------------------------------------------------------------------%
% 2- Remerciements.                                                    %
% ----------------------------------------------------------------------%

\remerciements
\selectlanguage{french}
Beaucoup de temps s'écoule entre le début et la fin d'une thèse. Notre
vie continue et plusieurs personnes nous marquent par leur arrivée, leur
passage, ou leur départ. Je commence ainsi mes remerciements en
accueillant la génération qui s'est ajoutée à ma vie et en saluant celle
qui m'a quittée. Liam et Alyssa, je vous aime de tout mon cœur et je
suis heureux d'avoir la chance d'être votre père. Je vous souhaite
d'être heureux et passionnés, et je vous promets d'être présent pour
rire ou pleurer avec vous des obstacles qui se présenteront sur votre
chemin. Grand-papa Charles, grand-maman Yvette, et grand-maman Thérèse :
votre présence a enrichi ma vie et je vous remercie pour tous les beaux
moments que nous avons partagés et qui ont contribué à façonner la
personne que je suis devenue. Je nourris l'espoir d'inspirer mes enfants
comme vous avez su m'inspirer.

Phil, ton enthousiasme infini et ton humour n'ont d'égal que ton fin
esprit stratégique et ta volonté de voir les gens qui t'entourent
réussir. Tu as su m'encourager à poursuivre mes idées les plus farfelues
en y contribuant systématiquement un soupçon de folie supplémentaire. Je
te remercie d'avoir cru en moi et de m'avoir encouragé à entreprendre
cette grande aventure; elle n'aurait pas été la même sans toi et je
serai toujours reconnaissant pour ces années passées en ta compagnie. Au
plaisir de poursuivre notre collaboration dans les prochaines années!

Dom, merci d'avoir accepté de travailler avec quelqu'un qui avait le
désir d'ajouter un soupçon de théorie dans sa vie sans réellement
comprendre ce que ça impliquait. À notre première rencontre, tu as
dessiné, sur ton tableau, comment tu envisagerais étudier la propagation
de perturbations à travers les communautés écologiques. Je n'ai rien
compris! Tu m'as permis d'ouvrir une porte sur un monde qui m'était
auparavant inconnu et au sein duquel j'ai appris à me plaire, si ce
n'est d'y être à mon aise! J'ai appris à réfléchir différemment et dans
ce processus j'ai aussi ajouté plusieurs cordes à mon arc qui auront,
j'en suis convaincu, une influence plus que significative pour le reste
de ma carrière. Merci!

Jean-Claude, tu étais mon patron, puis tu es devenu membre de mon comité
de thèse. À toutes ces étapes, je te remercie pour ton humour et ta
franchise absolue. Après la lecture de mon séminaire 1, tu m'as dit:
``Ouf! C'était lourd!''. Eh bien, je peux maintenant te dire, après
relecture en fin de projet, que tu avais raison! Merci pour cette
franchise et pour m'avoir fait confiance après ma maîtrise. J'apporterai
une bonne bouteille de scotch à mon prochain passage à Rimouski!

Je tiens également à remercier Isabelle et Fanny pour avoir accepté de
réviser ma thèse et pour les commentaires et discussions constructives
lors de la soutenance. J'espère avoir la chance de continuer ces
échanges dans le futur.

Le type de projet que j'ai entrepris est impossible sans l'apport d'un
nombre incalculable de collaborateurs qui agissent souvent dans l'ombre.
Plusieurs d'entre vous aviez un sourire en coin lorsque je vous exposais
mes idées de grandeur, mais vous vous êtes tout de même empressés de me
prêter main forte. Peter, Chris, Hugo, Simon, Pascal, Philippe, Clément,
Florian, Ben, Michel, Alfonso, Anne-Sophie, Andréanne, Simon, Guillaume,
Claudette, Deryck, Jean-François, Matthew, Geneviève, Hugues, Benoît,
Denis, Guy, Claude, Christian, et tout ceux que j'oublie ou dont
j'ignore l'existence, je vous remercie infiniment. Ce sont des
scientifiques comme vous qui nous permettront d'appliquer des approches
de gestion environnementale écosystémiques!

Kev, un merci particulier! Tu as incarné ce que j'envisage lorsque je
pense au mot ``\emph{collaboration}''. Tu as éclairé -- et agrémenté --
mon chemin théorique et informatique, et en cours de route tu es devenu
un ami inestimable. Peu importe ce que le futur me réserve, je suis
persuadé que tu y occuperas une place. En attendant, on pourra vider
Bowmore lorsque j'aurai terminé d'écrire!

Rémi, merci pour ta collaboration tout au long de ma thèse, mais surtout
merci pour toutes les conversations scientifiques et philosophiques que
nous avons eu, ensemble ou avec Peter et Ryan!

Charlotte, Marie, Elliot, Valérie et Laurie : merci pour le sel!

Jésica, je me sens choyé d'avoir débuté ma thèse à temps pour que nos
chemins se croisent. Ton écoute, ta présence et ta personnalité ont
ensoleillé mon parcours. Au plaisir de continuer de partager de beaux
moments dans le futur, et surtout de célébrer le mariage de nos enfants!

À tous les membres des laboratoires d'écologie benthiques, d'écologie
intégrative, de Québec Océan et de CHONe, je suis heureux d'avoir pu
partager ces années de science avec vous! J'ai déjà hâte au prochain 5 à
7, ou au prochain congrès!

\texttt{Steve,\ Nico\ et\ Kev\ (encore),\ mes\ amis\ geek,\ ce\ n’est\ peut-être\ pas\ vous\ qui\ m’avez\ amené\ à\ travailler\ sur\ les\ effets\ cumulatifs\ et\ les\ communautés,\ mais\ c’est\ très\ certainement\ vous\ qui\ m’avez\ aidé\ à\ m’outiller\ pour\ parvenir\ à\ mes\ fins.\ Lorsque\ l’envie\ de\ coder\ me\ reprendra\ -\/-\ sans\ doute\ dans\ quelques\ heures\ -\/-\ vous\ recevrez\ un\ message\ de\ ma\ part\ pour\ m’aider\ à\ trouver\ où\ j’ai\ bien\ pu\ faire\ une\ erreur!}

Maman, papa, Gilles, vous êtes des parents extraordinaires! Merci pour
vos encouragements et pour votre support. Vous me dites souvent que vous
êtes fiers de moi. Je peux vous dire en retour que je suis fier d'être
votre fils. Vous pouvez maintenant vous réjouir : à 34 ans, je quitte
enfin les bancs d'école; vous pourrez arrêter de me demander si je dois
étudier le soir ou la fin de semaine!

Cath. Nous nous sommes rencontrés tout juste après que j'aie décidé de
me lancer dans un projet de thèse. Tu n'avais aucune idée dans quoi tu
t'embarquais! Tu croyais commencer une relation avec un homme, mais
c'est un homme-étudiant que tu as eu. Pour ceux qui ont des doutes, ce
n'est pas la même chose! En commençant ma thèse, je me suis rapproché de
toi qui était à Québec\ldots{} en déménageant de Montréal vers Rimouski!
Tu es venue m'y rejoindre parce que tu croyais en nous, et j'ai fait la
même chose un an plus tard\ldots{} parce que Phil est déménagé à
l'Université Laval! Les joies d'être en couple avec un étudiant gradué!
Maintenant que cette aventure s'achève, ça sera à mon tour de te payer
un voyage en Asie! Je te remercie de m'avoir enduré et aimé pendant
toutes ces années! Je n'aurais pu imaginer une meilleure partenaire pour
me suivre, me supporter et m'encourager. Je ne pourrais surtout imaginer
une meilleure partenaire pour partager les joies -- et la folie --
d'être parents. Merci pour ta présence, pour être la mère de mes
enfants, et pour la personne que tu es. Mon rayon de soleil, je t'aime.


% [Cette page est facultative; l’éliminer si elle n’est pas utilisée. Les remerciements peuvent aussi être intégrés à l'avant-propos. C’est dans cette section que l’on remercie les personnes qui ont contribué au projet, les organismes ou les entreprises subventionnaires qui ont soutenu financièrement le projet.]



% ----------------------------------------------------------------------%
% 3- Avant-propos.                                                     %
% ----------------------------------------------------------------------%

\avantpropos
\selectlanguage{french}
Mon projet de doctorat s'insère dans le Canadian Healthy Oceans Network
(CHONe) du CRSNG. CHONe avait deux thèmes principaux de recherche, soit
les stratégies de conservation des écosystèmes marins ainsi que
l'identification des principaux stresseurs, incluant les effets
cumulatifs, qui altèrent la biodiversité marine et les fonctions et
services écosystémiques. Mon projet était à l'origine un projet de
maîtrise sur l'identification d'indicateurs de conditions benthiques à
l'échelle du Golfe du Saint-Laurent et devait mener à la caractérisation
des principales sources de stress au sein du Saint-Laurent. En fonction
de mes intérêts de recherche, nous avons développé ce projet de maîtrise
en un projet de doctorat qui vise également l'intégration des
interactions écologiques pour l'évaluation des effets cumulatifs directs
et indirects des stresseurs environnementaux sur les communautés
écologiques.

Mes travaux de thèse ont mené à la préparation de 5 articles
scientifiques, dont 3 sont publiés, 1 est en révision et le dernier est
en préparation. J'ai également contribué à la rédaction d'un chapitre de
livre sur l'évaluation des effets cumulatifs au sein du Système du
Saint-Laurent. J'ai également présenté mes travaux de recherche à de
multiples conférences à travers 11 présentations orales et 10 affiches
scientifiques.

Je veux reconnaître les nombreuses sources de support financier et
scientifique dont j'ai bénéficié tout au long de ma thèse. Je débute en
remerciant les organismes subventionnaires qui ont rendu mon projet
possible. Je remercie le Conseil de recherches en sciences naturelles et
en génie du Canada (CRSNG), qui m'a octroyé une bourse d'études
supérieures de doctorat (ES D), et le Fonds de recherche du Québec --
Nature et Technologies (FRQNT), qui m'a octroyé une bourse de doctorat
en recherche.

Je tiens également à remercier les multiples regroupements de recherche
dont j'ai été membre pendant ma thèse et qui m'ont fourni du support
scientifique et financier, ainsi que des expériences inoubliables et
enrichissantes autant d'un point de vue personnel que professionnel.
Merci à Québec Océan, le Centre de la science de la biodiversité (CSBQ),
le programme Computational Biodiversity Science and Services
(BIOS\(^2\)), Notre Golfe et Takuvik. Je remercie également le
regroupement CHONe et ses partenaires, soit le Département Pêches et
Océans Canada et l'INREST en tant que représentant du Port de Sept-Îles
et de la Ville de Sept-Îles. Je remercie également l'Observatoire Global
du Saint-Laurent (OGSL) pour son support technique dans l'établissement
de la plateforme \emph{eDrivers} et pour la collaboration prévue qui ira
au-delà de ma thèse.

Je remercie également divers ministères provinciaux et fédéraux qui ont
contribué au projet, que ce soit par du support financier, du temps
d'experts ou des données. Un merci particulier à Pêches et Océans
Canada, sans qui ce projet n'aurait été possible. Merci également au
Ministère de l'Environnement et de la Lutte contre les changements
climatiques (MELCC) et au Ministère de l'Agriculture, des Pêcheries et
de l'Alimentation du Québec (MAPAQ) du Gouvernement du Québec, au
Département Agriculture, Aquaculture et Pêches du Gouvernement du
Nouveau-Brunswick, au Département de Pêcheries et de l'Aquaculture du
Gouvernement de la Nouvelle-Écosse, et au Département de Pêcheries, de
Foresterie et d'Agriculture du Gouvernement de Terre-Neuve-et-Labrador.

Finalement, je tiens à remercier l'Université du Québec à Rimouski,
l'Université Laval et l'Université de Sherbrooke pour m'avoir accueilli,
fournit des locaux de travail et du support administratif.



% [Cette page est facultative; l’éliminer si elle n’est pas utilisée. L’avant-propos ne doit pas être confondu avec l'introduction. Il n’est pas d’ordre scientifique alors que l’introduction l’est. Il s’agit d'un discours préliminaire qui permet notamment à l'auteur d'exposer les raisons qui l'ont amené à étudier le sujet choisi, le but qu'il veut atteindre, ainsi que les possibilités et les limites de son travail. On peut inclure les remerciements à la fin de ce texte au lieu de les présenter sur une page distincte.]



% ----------------------------------------------------------------------%
% 4- Resume/Abstract                                                           %
% ----------------------------------------------------------------------%

\resume
\begin{singlespace}
Les écosystèmes sont de plus en plus soumis aux effets cumulatifs d'une
variété de stresseurs environnementaux en réponse à la demande
croissante en ressources naturelles et à l'intensification des
changements climatiques. Ces observations stimulent une demande
croissante pour les approches de gestion écosystémique et les
évaluations régionales des effets cumulatifs. Pourtant, une gestion
environnementale par silos centrée sur les évaluations par espèces et
stresseurs individuels demeure la norme. Cette absence d'approches
holistiques est particulièrement inquiétante pour la gestion d'espèces
exploitées ou en péril puisque leur dynamique, comme celle de toutes
espèces, est régie par le réseau d'interactions liant les espèces entre
elles au sein d'une communauté écologique et à travers lequel les effets
des stresseurs peuvent se propager indirectement. Cette disparité entre
les besoin d'approches holistiques et la pratique est partiellement
expliquée par un manque généralisé de connaissances théoriques des
effets de multiples stresseurs sur les communautés écologiques et par
des contraintes logistiques et méthodologiques. Dans ce contexte,
l'objectif général de ma thèse est d'évaluer les effets cumulatifs des
changements climatiques et des activités humaines sur les communautés
écologiques du Système du Saint-Laurent au Canada. Les hypothèses
générales de la thèse sont que les interactions influencent la
propagation indirecte et non-additive des effets de multiples stressors
à travers les communautés et que, en tant que tel, les interactions et
les stresseurs devraient être considérés conjointement au sein d'analyse
d'effets cumulatifs communauté-centrée.

La thèse est divisée en trois parties. Dans la première partie (chapitre
1), je conceptualise la propagation des stresseurs environnementaux à
travers les réseaux trophiques et j'explore théoriquement comment des
stresseurs simulés affectent des motifs à trois espèces et des
communautés du Système du Saint-Laurent. Nous trouvons que négliger les
interactions écologiques sous-estime systématiquement les effets des
stresseurs et que les effets synergiques et antagonistes sont fréquents
à travers les interactions. À l'échelle des réseaux, nous trouvons que
les prédateurs apicaux sont négativement affectés, alors que les
méso-prédateurs bénéficient des effets des stresseurs dans le Système du
Saint-Laurent. Par contre, la sensibilité des espèces dépend de la
structure des réseaux trophiques. Le travail théorique proposé au
chapitre 1 permet de valider les hypothèses générale de la thèse et
offre un cadre accessible et appuyé par la théorie écologique pour
inclure les interactions écologiques à l'évaluation des effets
cumulatifs.

La deuxième partie de ma thèse aborde des défis logistiques et
méthodologiques pour l'évaluation des effets cumulatifs des changements
climatiques et des activités humaines sur les communautés du Système du
Saint-Laurent. Au chapitre 2, j'aborde le défi de caractériser les
interactions écologiques au sein de systèmes où peu de données sont
disponibles. Je présente une nouvelle méthode d'apprentissage non
supervisée pour prédire les interactions binaires à partir de la
proximité taxonomique entre espèces et d'une collection d'interactions
empiriques connues entre espèces marines. Les résultats suggèrent que
les interactions écologiques peuvent être prédites avec précision, ce
qui pourrait promouvoir leur utilisation pour la gestion
environnementale. Au chapitre 3, j'identifie des enjeux environnementaux
et suggère des priorités de recherche et de gestion pour promouvoir les
évaluations d'effets cumulatifs et la gestion écosystémique dans le
Système du Saint-Laurent. Au chapitre 4, je caractérise la distribution
et l'intensité de stresseurs environnementaux issus des activités
humaines et des changements climatiques dans le Système du
Saint-Laurent. À partir de collaborations, d'initiatives
environnementales existantes et de portails de données ouvertes, 22
stresseurs d'origine côtière, du climat, de la pêche et du trafic
maritime ont été caractérisés. Les résultats du chapitre 4 démontrent
que les stresseurs sont répandus partout dans le Système du
Saint-Laurent et que l'estuaire, la gyre d'Anticosti, et les milieux
côtiers sont particulièrement exposés aux stresseurs.

Dans la troisième partie de la thèse (chapitre 5), je présente une
évaluation des effets cumulatifs sur 193 espèces du Système du
Saint-Laurent à partir d'une nouvelle approche communauté-centrée qui
considère les interactions écologiques et les effets indirects.
L'approche proposée s'appuie sur le cadre théorique présenté au chapitre
1 et utilise les résultats et méthodes des chapitres 2, 3 et 4. Je
compare cette approche à une approche conventionnelle espèce-centrée
pour exposer des propriétés émergentes provenant des interactions
écologiques et des effets sur des espèces qui seraient normalement
ignorés. Pour certaines espèces, considérer les interactions écologiques
pourraient être l'unique moyen d'évaluer les effets des stresseurs
environnementaux. Les poissons et les mammifères marins sont
particulièrement susceptibles aux effets indirects de tous les types de
stresseurs considérés; ceci est un contraste marqué avec le nombre
limité de stresseurs qui les affectent directement.

En alliant théorie, gestion environnementale et bio-informatique, ma
thèse montre que les interactions écologiques sont un élément clé à
considérer pour l'évaluation des effets de multiple stresseurs et
propose une approche accessible pour les intégrer aux évaluations
d'effets cumulatifs. Mes résultats sont particulièrement pertinents pour
la gestion d'espèces exploitées et en péril, pour qui nous ignorons
potentiellement des risques importants en négligeant les effets
indirects provenant des interactions entre espèces. Développer ces
capacités holistiques est essentiel en vue d'opérationnaliser un mode de
gestion environnementale écosystémique.

\begin{quote}
Mots clés: effets cumulatifs, stresseurs environnementaux, effets
indirects, interactions biotiques, effets non-additifs, communautés
écologiques, réseaux trophiques, Système du Saint-Laurent, activités
humaines, changements climatiques.
\end{quote}

  % [Le résumé en français doit présenter en 350 mots maximum pour un mémoire et en 700 mots pour une thèse : (1) le but de la recherche, (2) les sujets étudiés, (3) les hypothèses de travail et la méthode utilisée, (4) les principaux résultats et (5) les conclusions de l'étude ou de la recherche.]

\end{singlespace}
\cleardoublepage


\abstract
\begin{singlespace}
With demands for natural resources increasing alongside populations and
the effects of climate change intensifying, ecosystems worldwide are
increasingly burdened with the cumulative effects of a vast array of
environmental stressors. These observations stimulate a growing demand
for ecosystem-based approaches and regional cumulative effects
assessments. Yet environmental management still overwhelmingly operates
in silos, focusing instead on single-stressor and single-species
assessments. This is particularly concerning for exploited and
endangered species whose dynamics, like that of all species, is driven
by the network of interactions structuring ecological communities and
through which the effects of stressors can propagate indirectly. A
general lack of theoretical understanding of the effects of multiple
stressors on ecological communities, and methodological and logistical
constraints explain part of this discrepancy. In this context, the
general objective of this thesis is to evaluate the cumulative effects
of climate change and human activities on the ecological communities of
the St.~Lawrence System in eastern Canada. The general hypotheses
guiding the thesis are that species interactions influence the indirect
and non-additive spread of the effects of multiple stressors through
communities and that, as such, species interactions and stressors should
be considered together in network-scale cumulative effects assessments.

The thesis is divided in three parts. In the first part (chapter 1), I
conceptualize how stressors propagate through food webs and explore how
they affect simulated 3-species motifs and food webs of the Canadian
St.~Lawrence System. We find that overlooking species interactions
invariably underestimates the effects of stressors, and that synergistic
and antagonistic effects through food webs are prevalent. At the scale
of food webs, we find that apex predators generally were negatively
affected and mesopredators benefited from the effects of stressors in
the St.~Lawrence System, but that species sensitivity is dependent on
food web structure. The theoretical simulations proposed in chapter 1
validate the general hypotheses of the thesis and provide an accessible
and theory-grounded framework for the inclusion of species interactions
in cumulative effects assessments.

The second part of the thesis addresses logistical and methodological
challenges for the cumulative effects assessment of climate change and
human activities on communities of the St.~Lawrence System. In chapter
2, I address the challenge of characterizing ecological interactions in
data-deficient ecosystems. I present a new unsupervised machine learning
method to predict interactions between any given set of taxa, given
pairwise taxonomic proximity and known consumer and resource sets
available through various open-data portals. Results from chapter 2
suggest that ecological interactions can be predicted with high
accuracy, which could promote their use for environmental management. In
chapter 3, I identify environmental issues and suggest research and
management priorities to promote cumulative effects assessment and
ecosystem-based management in the St.~Lawrence System. In chapter 4, I
characterize the distribution and intensity of environmental stressors
arising from human activities and climate change in the St.~Lawrence
System. Through collaborations, existing environmental initiatives and
open data portals, I gathered data-based indicators for 22 coastal,
climate, fisheries, and marine stressors. Results from chapter 4 show
that stressors are widespread and that coastal areas and the Estuary,
Anticosti Gyre, and coastal areas are particularly exposed to cumulative
exposure and hotspots.

In the third part of the thesis (chapter 5) I present a new
network-scale approach to assess cumulative effects that explicitly
considers ecological interactions and indirect effects. The approach is
built on the framework presented in chapter 1 and uses results and
methods from chapters 2, 3 and 4. I compare our approach to a
conventional species-scale assessment to expose transgressive properties
arising from species interactions and uncover cumulative effects to
species that would otherwise be overlooked. Fishes and marine mammals
appear particularly prone to indirect effects from all types of
stressors; this contrasts considerably with the limited number of
stressors affecting them directly. For certain species, considering
interactions may even be the only means of assessing the effects of
stressors.

My research findings show that the intricacies of ecological communities
are key to assess the direct and indirect effects of multiple stressors
on species and how ecological interactions can be explicitly considered
in cumulative effects assessment. This is particularly relevant to the
management of exploited and endangered species for which we may
currently ignore significant threats by overlooking the less obvious yet
no less significant effects arising from species interactions. My thesis
also promotes a systems mindset that could be instrumental in fulfilling
the promise of ecosystem-based management.

\begin{quote}
Keywords: cumulative effects, environmental stressors, indirect effects,
biotic interactions, non-additive effects, ecological network, food web,
St.~Lawrence System, human activities, climate change.
\end{quote}


  % [L'abstract doit être une traduction anglaise fidèle et grammaticalement correcte du résumé en français.]

\end{singlespace}
\cleardoublepage




% ----------------------------------------------------------------------%
% 5- Table des matières.                                               %
% ----------------------------------------------------------------------%

\tabledesmatieres



% ----------------------------------------------------------------------%
% 6- Liste des tableaux.                                               %
% ----------------------------------------------------------------------%

\listedestableaux

% ----------------------------------------------------------------------%
% 7- Table des matières.                                               %
% ----------------------------------------------------------------------%

\listedesfigures

% ----------------------------------------------------------------------%
% 8- Liste des abréviations (optionnel).                               %
% ----------------------------------------------------------------------%

\listeabrev
\begin{liste}

\item[DOI]~: \textit{Digital Object Identifier}; identifiant numérique d'objet.

\item[GIEC]~: Groupe d'experts Intergouvernemental sur l'Évolution du Climat.

\item[IPBES]~: \textit{Intergovernmental Science-Policy Platform on Biodiversity and Ecosystem Services}; Plateforme intergouvernementale sur la biodiversité et les services écosystémiques.

\end{liste}



% ----------------------------------------------------------------------%
% 9- Liste des symboles (optionnel).                                   %
% ----------------------------------------------------------------------%

% \listesymboles
% \begin{liste}
% \item[SYMBOLE 1] Ceci est la définition du symbole 1.
%
% \item[SYMBOLE 2] Ceci est la définition du symbole 2.
%
% \item[SYMBOLE 3] Ceci est la définition du symbole 3.
% \end{liste}

% ----------------------------------------------------------------------%
% Fin des liminaires.                                                  %
% ----------------------------------------------------------------------%

\cleardoublepage
