\chapter{L’évaluation des impacts cumulés dans l’estuaire et le golfe du Saint-Laurent : vers une planification systémique de l’exploitation des ressources}
\label{chap1}

\section{Résumé en français du premier article}

\subsection{Contexte scientifique}

\subsection{Publication associée}

\subsection{Traduction du résumé de l'article publié}

The intensification of human activity in the Estuary and Gulf of St. Lawrence (Canada) imposes the need for a systematic planning approach for the use of marine resources. There is, however, currently no regional cumulative impact assessment for the St. Lawrence. Many of the human activities in this area (e.g., shipping, fisheries and aquaculture) impose environmental threats (e.g., habitat destruction) that may jeopardize ecosystem structure and function. Increasingly, these threats are overlapping spatially, which induces synergies causing unpredictable non-linear effects on ecosystems. These effects are still poorly understood and consequently neglected in environmental impact assessments, which remain focused on single species or sectors, and on the approval of specific projects. To efficiently evaluate cumulative impacts in the St. Lawrence, it will be important to: 1) improve our knowledge concerning the impacts of multiple threats to ecosystems; 2) improve the accessibility to, and the applicability of, cumulative impact tools; 3) identify relevant human and environmental indicators of cumulative impacts; 4) create a data sharing, and human impact and environmental monitoring protocol; and 5) develop an adaptive management approach for the St. Lawrence. Systematic planning of the use of natural resources in the St. Lawrence will require an integrated overview of the structure and function of its ecosystems, and of the sources of stresses affecting them. Such an approach will only be feasible once the necessary infrastructures and tools for ecosystem-based management of the area have been developed.

\textit{Les sections suivantes sont celles de l’article publié.}
