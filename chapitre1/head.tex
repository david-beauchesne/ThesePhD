\chapter{À propos de la sensibilité des réseaux trophiques aux effets de multiples stresseurs}
\label{chap1}

\section{Résumé}

Évaluer les effets de multiples stresseurs sur les écosystèmes est de plus en plus important en raison des changements globaux. Le rôle des interactions écologiques pour la propagation des effets des stresseurs, bien que largement reconnue, n’a toujours pas été formellement explorée. Nous avons conceptualisé comment les stresseurs se propagent à travers les réseaux trophiques et nous avons exploré comment les stresseurs affectent des motifs à 3 espèces simulés et des réseaux trophiques du Système du Saint-Laurent canadien. Nous trouvons que négliger les interactions écologiques sous-estime systématiquement les effets des stresseurs et que les effets synergiques et antagonistes sont fréquents à l’échelle des réseaux trophiques. Nous trouvons également que différents types d’interactions influencent la sensibilité des espèces aux effets des stresseurs : les espèces impliquées dans des interactions omnivores et des chaînes tri-trophiques sont particulièrement sensible (points d’entrée faibles) et susceptibles aux effets synergiques (amplificateurs biotiques) et antagonistes (tampons biotiques). Finalement, nous trouvons que les prédateurs apicaux sont négativement affectés, alors que les méso-prédateurs bénéficient des effets des stresseurs dû à leur position trophique respective dans le Système du Saint-Laurent. Toutefois, la sensibilité des espèces dépend de la structure des réseaux trophiques. Par notre conceptualisation des effets de multiples stresseurs sur les réseaux trophiques, nous rapprochons la théorie de la pratique et démontrons que la complexité de la structure des communautés écologiques est essentielle à une évaluation des effets nets des stresseurs sur les espèces.

Ce premier article, intitulé \textit{"On the sensitivity of food webs to multiple stressors"} a été corédigé par moi-même, Kevin Cazelles, Laura Dee, Philippe Archambault et Dominique Gravel. Il est actuellement en révision pour resoumission dans la revue \textit{Ecology Letters} en tant qu'article de type \textit{Ideas and Perspectives}. L'article est actuellement disponible en version \textit{preprint} (https://doi.org/10.22541/au.159621485.58777803). Tous les auteurs ont contribué à l'élaboration des objectifs de l'article. En tant que premier auteur, j'ai été en charge de la conceptualisation, des simulations, des analyses, des figures et j'ai dirigé la rédaction de l'article. Kevin Cazelles a contribué à ces étapes. Tous les auteurs ont contribué aux bases de données, aux analyses et à l'écriture de l'article selon leur expertise respective et ont contribué à la révision de l'article. Les résultats issus de cet article ont été présentés en version abrégée lors du \textit{Forum québécois en sciences de la mer} à Rimouski (Canada) à l'automne 2019. \linebreak[4]

\begin{singlespace}
Beauchesne, D., Cazelles, K., Archambault, P., Dee, L., Gravel, D., 2020. On the sensitivity of food webs to multiple stressors (preprint). Preprints. https://doi.org/10.22541/au.159621485.58777803
\end{singlespace}

\textit{Les sections suivantes sont celles de l’article en révision.}
